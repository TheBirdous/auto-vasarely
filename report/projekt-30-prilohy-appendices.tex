% Tento soubor nahraďte vlastním souborem s přílohami (nadpisy níže jsou pouze pro příklad)

% Pro kompilaci po částech (viz projekt.tex), nutno odkomentovat a upravit
%\documentclass[../projekt.tex]{subfiles}
%\begin{document}

% Umístění obsahu paměťového média do příloh je vhodné konzultovat s vedoucím
%\chapter{Obsah přiloženého paměťového média}


\chapter{Obsah CD}

\begin{itemize}
    \item \textbf{doc/} je složka, ve které se nachází dokumentace modulů aplikace.
    \item \textbf{excel/} obsahuje prezentační materiály pro konferenci Excel@FIT 2023.
    \item \textbf{profiling/} je adresář se soubory obsahující statistické údaje využité při ladění rychlosti aplikace.
    \item \textbf{report/} je složka obsahující zdrojový tvar této písemné zprávy.
    \item \textbf{resources/} je složka obsahující zdroje využité při první fázi testování.
    \item \textbf{src/} je adresář se zdrojovými soubory aplikace.
    \item \textbf{test/} je složka s testovacími skripty.
    \item \textbf{test\_report/} je složka se zdroji využitými v druhé fázi testování, která je zahrnuta v technické zprávě.
    \item \textbf{LICENSE} představuje soubor obsahující licenční informace k programu.
    \item \textbf{README.md} je soubor obsahující návod na spuštění aplikace.
    \item \textbf{xdohna48.pdf} obsahuje tuto technickou zprávu.
\end{itemize}


% Pro kompilaci po částech (viz projekt.tex) nutno odkomentovat
%\end{document}
